\documentclass[a4paper,11pt,twoside]{article}
%\documentclass[a4paper,11pt,twoside,se]{article}

\usepackage{UmUStudentReport}
\usepackage{verbatim}   % Multi-line comments using \begin{comment}
\usepackage{courier}    % Nicer fonts are used. (not necessary)
\usepackage{pslatex}    % Also nicer fonts. (not necessary)
\usepackage[pdftex]{graphicx}   % allows including pdf figures
\usepackage{listings}
\usepackage{pgf-umlcd}
\usepackage{blindtext}
\usepackage{rotating}
\usepackage{enumitem}
%\usepackage{lmodern}   % Optional fonts. (not necessary)
%\usepackage{tabularx}
%\usepackage{microtype} % Provides some typographic improvements over default settings
%\usepackage{placeins}  % For aligning images with \FloatBarrier
%\usepackage{booktabs}  % For nice-looking tables
%\usepackage{titlesec}  % More granular control of sections.

% DOCUMENT INFO
% =============
\department{Department of Computing Science}
\coursename{Development of Mobile Appliations 7.5 p}
\coursecode{5DV155}
\title{`Mealpricer' - Free of Choice App Project}
\author{Lorenz Gerber ({\tt{dv15lgr@cs.umu.se}} {\tt{lozger03@student.umu.se}})}
\date{2017-08-21}
%\revisiondate{2016-01-18}
\instructor{Johan Eliasson / Jonathan Westin}


% DOCUMENT SETTINGS
% =================
\bibliographystyle{plain}
%\bibliographystyle{ieee}
\pagestyle{fancy}
\raggedbottom
\setcounter{secnumdepth}{2}
\setcounter{tocdepth}{2}
%\graphicspath{{images/}}   %Path for images

\usepackage{float}
\floatstyle{ruled}
\newfloat{listing}{thp}{lop}
\floatname{listing}{Listing}



% DEFINES
% =======
%\newcommand{\mycommand}{<latex code>}

% DOCUMENT
% ========
\begin{document}
\lstset{language=C}
\maketitle
\thispagestyle{empty}
\newpage
\tableofcontents
\thispagestyle{empty}
\newpage

\clearpage
\pagenumbering{arabic}

\section{Introduction}
The aim was to come up with a good idea for a reasonable sized mobile application
that should then be implemented. Reasonable size here was that implementation should
be possible in about 60 hours of work (1.5 week).

During the course, there were some propositions of possible applications where one
was a cooking recipt app with shopping list functionality. I like to cook and I am
usually on a tight budget. Further, I actually like to guess or calculate the price
of all kind of stuff I see around me. Hence, from the cooking receipt application I got
inspired to develop `Mealpricer', an app to calculate the fractional costs of meals.

Generally, wile pondering about potential projects, I was convinced that I did not
want to implement a game as this would involve mostly general java programming. My
aim was to find a project that would give me the possibility to improve on my
general crafts for navigating the android framework. In my opinion, the most typical
mobile application is data centric. It presents some sort of data in lists. Currently,
to make a professional looking android app, I think it is indispensable to master the
elements of material design. Therefore, another aim for me was to incorparate
as much material design as possible in my project. So far, I have not spent much
time with GUI programming in general, there I wanted to develop more in this field.


\section{General Description and Target Group}
The envisioned and implemented android app with the name `Mealpricer' is
targeted towards indviduals that are looking for a tool to quick and easily determine
the fractional cost of a specific meal.

I checked out the Google Play Store for similar apps. There were a few with a
smiliar idea. There I also realized the potential behind the
concept: Calculating fractional costs of a meal from bulk ingredient prices and
ingredient amounts is actually a very common application for restaurants to price
their meals.

So besides individuals, I think this application could make a good tool for
food and catering professionals to quickly recalculate meal prices while
shopping for ingredients.

\section{PlayStore Description}
\textit{MealPricer is an application that helps you to keep track of how much individual
meals that you prepare cost. Whether you are a student on a tight budget or a
restaurant owner that wants to calculate the fractional costs of his offerings
from bulk products, MealPricer will help you in the most simple way to keep track
of it.}

\section{Target Course Grade}
I aim for a 'VG' and below follows some reasoning around it.

I have spent quite a bit more time on the app than the indicated
target of 60 hours. I am aware that for a senior android developer, such an
app could be developed in a fraction of that time. However, I think it is
also understood that this is an introduction course to the android framework, hence
additional time is justified.

Generally, I am pretty happy with the result of my application. I am aware that
the app has a rather simple concept with an easy to implement data model. But this
was also by purpose as mentioned earlier: My aim was to spend as much time as
possible with actual android framwork programming and not general Java. I think that
the final product looks appealing and can be used practically. I spent quite some
time on drafting the initial flow of the app and looking up which concepts of
material design that would make for a good user experience.



\section{Aspects on Security and Ethics}
\section{User Guide - How To}
\section{Application Architecture}
\section{Selected Discussion Topics}

\subsection{Embedding Fragments statically/dynamically in Activity}
Trying to use the provided templates for google material designed activities and
fragments, I run into the situation where fragments where by default embedded into
activities by instantiating them in XML. This works fine when the Activity/Fragment
combo in question does not need to obtain extras on creation. In many cases however,
for example when the Activity/Fragment combo represents the detail view after a
list activity, the ID of the chosen list item has to be obtained. Hence, the given
template code had to be modified for dynamic instantiation of the fragment from
within the activity.
During planning for the above modification, it was noted that there are at least
three single activity / single fragment views in the current application. According
to the example in the course book, this could make up for an `Abstract'
`SingleFragmentActivity' class. However, it was decided against this due to
implications on the material design elements: The Appbar and Floating Action Button
are kept in the activity, while the content layout is in the fragment. But the
three single fragment activities don't share exactly the same setup for appbar
and Floating Action Buttons. Hence another layer of abstraction would be needed that
would make things overly verbose.

\subsection{Deciding on the Persistance Model} It was from the beginning decided
to store the data in SQLite. It was needed to decide how to store data from the
UI on for example orientation changes. Currently, in the IngredientChooser, when
an ingredient is not selected, the value will not persist when the device
changes orientation. To come to a somehow fluent user experience, it was decided
to connect the checkbox for selected with the textchange listener of the entry
boxes. Hence, when text is entered, the selected textbox is activated. In the
same sense, to improve the user experience, unselecting the checkbox will
automatically also empty the entry fields. As such, it is no longer possible to
have entered text without selection active. Hence, on orientation change, all
data will be stored in the DB and read back on resume.

\subsection{Deciding on Ingredient Chooser UI}
Showing zero's or not. Showing a search bar. Sorting alphabetically.

\subsection{In InigredientChooser, correct initial focus}
On ActivityStartup, always the first field get's focus before data is loaded.
The onFocus method then determines that there is no data, hence it writes/overwrites
directly a zero to the database. The solution was to add a dummy focus on the
title element in the XML file.


\addcontentsline{toc}{section}{\refname}
\bibliography{references}

\end{document}
