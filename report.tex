\documentclass[a4paper,11pt,twoside]{article}
%\documentclass[a4paper,11pt,twoside,se]{article}

\usepackage{UmUStudentReport}
\usepackage{verbatim}   % Multi-line comments using \begin{comment}
\usepackage{courier}    % Nicer fonts are used. (not necessary)
\usepackage{pslatex}    % Also nicer fonts. (not necessary)
\usepackage[pdftex]{graphicx}   % allows including pdf figures
\usepackage{listings}
\usepackage{pgf-umlcd}
\usepackage{blindtext}
\usepackage{rotating}
\usepackage{enumitem}
%\usepackage{lmodern}   % Optional fonts. (not necessary)
%\usepackage{tabularx}
%\usepackage{microtype} % Provides some typographic improvements over default settings
%\usepackage{placeins}  % For aligning images with \FloatBarrier
%\usepackage{booktabs}  % For nice-looking tables
%\usepackage{titlesec}  % More granular control of sections.

% DOCUMENT INFO
% =============
\department{Department of Computing Science}
\coursename{Development of Mobile Appliations 7.5 p}
\coursecode{5DV155}
\title{`Mealpricer' - Free of Choice App Project}
\author{Lorenz Gerber ({\tt{dv15lgr@cs.umu.se}} {\tt{lozger03@student.umu.se}})}
\date{2017-08-21}
%\revisiondate{2016-01-18}
\instructor{Johan Eliasson / Jonathan Westin}


% DOCUMENT SETTINGS
% =================
\bibliographystyle{plain}
%\bibliographystyle{ieee}
\pagestyle{fancy}
\raggedbottom
\setcounter{secnumdepth}{2}
\setcounter{tocdepth}{2}
%\graphicspath{{images/}}   %Path for images

\usepackage{float}
\floatstyle{ruled}
\newfloat{listing}{thp}{lop}
\floatname{listing}{Listing}



% DEFINES
% =======
%\newcommand{\mycommand}{<latex code>}

% DOCUMENT
% ========
\begin{document}
\lstset{language=C}
\maketitle
\thispagestyle{empty}
\newpage
\tableofcontents
\thispagestyle{empty}
\newpage

\clearpage
\pagenumbering{arabic}

\section{Introduction}
\section{General Description and Target Group}
\section{PlayStore Description}
\section{Target Course Grade}
\section{Aspects on Security and Ethics}
\section{User Guide - How To}
\section{Application Architecture}
\section{Selected Discussion Topics}
\subsection{Embedding Fragments statically/dynamically in Activity}
Trying to use the provided templates for google material designed activities and
fragments, I run into the situation where fragments where by default embedded into
activities by instantiating them in XML. This works fine when the Activity/Fragment
combo in question does not need to obtain extras on creation. In many cases however,
for example when the Activity/Fragment combo represents the detail view after a
list activity, the ID of the chosen list item has to be obtained. Hence, the given
template code had to be modified for dynamic instantiation of the fragment from
within the activity.
During planning for the above modification, it was noted that there are at least
three single activity / single fragment views in the current application. According
to the example in the course book, this could make up for an `Abstract'
`SingleFragmentActivity' class. However, it was decided against this due to
implications on the material design elements: The Appbar and Floating Action Button
are kept in the activity, while the content layout is in the fragment. But the
three single fragment activities don't share exactly the same setup for appbar
and Floating Action Buttons. Hence another layer of abstraction would be needed that
would make things overly verbose.  


\addcontentsline{toc}{section}{\refname}
\bibliography{references}

\end{document}
